\documentclass[sw]{iosart2x}

%%%%%%%%%%% Put your definitions here
\usepackage{hyperref}
\usepackage[1000]{aurl}
\daurl{swo}{http://www.ebi.ac.uk/swo/}
\daurl{}{http://hitontology.eu/ontology/}
\usepackage{graphicx}
\usepackage{microtype}
\usepackage{cleveref}
\usepackage{siunitx}
\usepackage{booktabs}
\usepackage{tabulary}
\usepackage{csquotes}
\usepackage[nolist]{acronym}
%\newcommand{\citet}{\cite}% citet is not defined without natbib
%\newcommand{\citep}{\cite}% citep is not defined without natbib
\newcommand{\hito}[1]{\aurl{}{#1}}

\begin{acronym}[SPARQL]
\acro{HITO}{Health IT Ontology}
\acro{SPARQL}{SPARQL}
\acro{SWO}{The Software Ontology}
\acro{DOAP}{Description of a Project}
\acro{MEDFLOSS}{Medical Free/Libre Open Source Software}
\end{acronym}
%%%%%%%%%%% End of definitions

\pubyear{2021}
\volume{0}
\firstpage{1}
\lastpage{1}

\begin{document}

\begin{frontmatter}
%\pretitle{}
\title{HITO Ontology and Knowledge Base: Medical Free/Libre and Open Source Software Products as Linked Open Data}
\runtitle{HITO Ontology and Knowledge Base}
%\subtitle{}

\begin{aug}
\author[A]{\inits{N.}\fnms{Name1} \snm{Surname1}\ead[label=e1]{first@somewhere.com}%
\thanks{Corresponding author. \printead{e1}.}}
%\author[B]{\inits{N.N.}\fnms{Name2 Name2} \snm{Surname2}\ead[label=e2]{second@somewhere.com}}
%\author[B]{\inits{N.-N.}\fnms{Name3-Name3} \snm{Surname3}\ead[label=e3]{third@somewhere.com}}
\address[A]{Department first, \orgname{University or Company name}, Abbreviate US states, \cny{Country}\printead[presep={\\}]{e1}}
%\address[B]{Department first, \orgname{University or Company name}, Abbreviate US states, \cny{Country}\printead[presep={\\}]{e2,e3}}
\end{aug}

%\begin{review}{editor}
%\reviewer{\fnms{First} \snm{Editor}\address{\orgname{University or Company name}, \cny{Country}}}
%\reviewer{\fnms{Second} \snm{Editor}\address{\orgname{First University or Company name}, \cny{Country}
%    and \orgname{Second University or Company name}, \cny{Country}}}
%\end{review}
%\begin{review}{solicited}
%\reviewer{\fnms{First} \snm{Solicited reviewer}\address{\orgname{University or Company name}, \cny{Country}}}
%\reviewer{\snm{anonymous reviewer}}
%\end{review}
%\begin{review}{open}
%\reviewer{\fnms{First} \snm{Open Reviewer}\address{\orgname{University or Company name}, \cny{Country}}}
%\end{review}

\begin{abstract}
Medical software is a large global market and the choice of software for each application system is criticial for the stability, efficiency and capability of a hospital information systems.
However it is time consuming and error prone to compare those products using official documents such as advertisements, presentations and web sites.
Providing transparent and structured data about medical software, including open free/libre source products, helps setting up new hospitals, especially in resource-poor environments.
The open data initiative Medfloss.org provide semi-structured descriptions about software descriptions, however it does not cover some aspects such as features and the data is not available as a structured format.
To alleviate these shortcomings, we present HITO, an ontology for and a knowledge graph of software products.
%Following the best-of-breed strategy, hospitals can choose amongst the best suited software products for each of its many application systems.

%At the start, HITO defined everything as new classes and properties.
%However this didn't take advantage of the LOD cloud and was thus changed.
%Replacing self-defined classes with links to existing ones reduces development effort, improves maintainability and allows deeper exploration.
%We investigate ontologies that are close to HITO for their suitability as replacements for HITO classes.
\end{abstract}

\begin{keyword}
\kwd{health information systems}
\kwd{medical software}
\kwd{open source software}
\kwd{knowledge graph}
\kwd{open data}
\kwd{ontology}
\end{keyword}

\end{frontmatter}

%%%%%%%%%%% The article body starts:
\section{Introduction}\label{sec:introduction}

%\subsection{Subject}
The aim of the \acf{HITO} is to enable systematic descriptions of application systems and software products in health IT.
In the initial draft of \ac{HITO}~\citep{towardsprecise}, classes for related concepts such as operating systems, licenses and programming languages lie in the HITO namespace.
We consider candidate sets of individuals from the LOD cloud for each of those concepts.
Based on the data quality, we select and refine sets of invididuals from DBpedia~\citep{dbpedia} and \ac{SWO}~\citep{swo}.
We allow users to generate new instances and to select among those individuals by developing and offering a user interface to generate new instances of \ac{MEDFLOSS} products.
This is joined with converted \ac{MEDFLOSS} products from \url{https://www.medfloss.org}.

%\subsection{Problem}
Not all \ac{MEDFLOSS} domain experts are fluent in Semantic Web technologies.
Creating new instances in an RDF syntax such as Turtle is a challenge for them, that a user interface can alleviate.
Relying purely on self-defined classes to describe \ac{MEDFLOSS} products is problematic however, because even with a user interface our frequent use case of creating a new instance is still time-consuming.
There are many related individuals, such as programming languages, licenses and operating systems, that need to be created as well.
Furthermore, letting users create those attribute individuals risks data quality problems such as duplicates, inconsistencies and missing or incorrect values.
Finally, self-defined individuals would lack updated values and new addition after the end of the research project.
% small hospitals and those in developing countries are restricted in their choice of software products due to their high cost
\subsection{Motivation}
Generating high-quality sets of individuals from existing sources and presenting them in a UI has multiple benefits:
(1) Large time-savings when creating instances of \ac{MEDFLOSS} products.
(2) Improved data quality in the dimensions~\citep{dataquality} syntactic validity, semantic accuracy, consistency, completeness, trustworthiness, understandability timelyness and interoperability 
(3) Improved intuitiveness

\section{Related Work}
\begin{itemize}
\item DBpedia, the central knowledge base of the LOD cloud, contains many properties and classes related to software products. However the data quality is lacking.
\item SWO (The Software Ontology)~\citep{}
\item DOAP (Description of a Project)\footnote{\url{http://usefulinc.com/ns/doap}}
\end{itemize}

\section{HITO Ontology}

\begin{figure}[h]
\includegraphics[width=\textwidth]{img/hito-diagram.pdf}
\caption{fig:hito-diagram}
\label{The HITO ontology for representing properties of studies and software products, such as features and enterprise functions.}
\end{figure}

The HITO ontology, introduced in \citet{hitometh,towardsprecise}, describes software products and their properties as well as research studies about such software products.
see \cref{fig:hito-diagram}
The core classes related to \hito{SoftwareProduct} are:
\begin{itemize}
\item \hito{EnterpriseFunction}
\item \hito{Feature}
%\item UserGroup
\item \hito{ApplicationSystem}, and
\item \hito{OrganizationalUnit}
\end{itemize}

An enterprise function of a hospital is a ... source ...
A feature of a software product is a ...
An application system is a ... source ...
An organisational units, such as a laboratory or a ..., helps ... .

Each of those classes is referenced in a product description represented by a \emph{citation}, which can be linked to a \emph{classified} entry of a \emph{catalogue}.
For example, the software product \hito{Bahmni} references a feature \enquote{...} represented by the citation resource \hito{BahmniFeat...}, which is linked to the classified feature \hito{Bb...} of the catalogue \hito{BbFeatureCatalogue} extracted from \citet{bb}.

\section{HITO Knowledge Base}

\section{Architecture}
\Cref{fig:architecture} shows the architecture with the SPARQL endpoint as the central API for all user interfaces:
\begin{itemize}
\item RDF browser
\item faceted search
\item graph visualization
\end{itemize}


As RDF browser we use LodView~\citep{lodview}.

\begin{figure}[h]
%\includegraphics[width=\textwidth]{img/hito-architecture.pdf}
\caption{fig:hito-architecture}
\label{The architecture of HITO}
\end{figure}

\section{Services}
\section{Data Collection}
Software product data entries are held in a relational database and are created, updated and deleted in a web application using the Python Flask AppBuilder based on the SQLAlchemy framework.

\subsection{Software License}

\begin{tabular}{llcc}
\toprule
Ontology	&property			&range	&instances\\
\midrule
DBpedia		&\aurl{dbo}{license}		&---	&---\\
SWO		&\aurl{swo}{SWO\_0000002}	&	&\\
DOAP		&\\
\bottomrule
\end{tabular}

\subsection{Operating System}

\begin{tabular}{llcc}
\toprule
Ontology	&property			&range	&instances\\
\midrule
DBpedia		&\aurl{dbo}{operatingSystem}	&---	&---\\
YAGO		&				&	&\\
SWO		&\\
DOAP		&\\
\bottomrule
\end{tabular}

\begin{tabular}{llcccc}
\toprule
Ontology	&class					&instances	&missing			&wrong\\
\midrule
YAGO		&\aurl{yago}{OperatingSystem106568134}	&320		&winxp, arch linux		&many\\
YAGO		&\aurl{yago}{WikicatOperatingSystems}	&		&winxp, arch linux		&many\\
SWO		&\\
DOAP		&\\
\bottomrule
\end{tabular}


\subsection{Programming Language}

\section{Other}
%\begin{}
link to studies, which in turn are systematically described by our ontology,
links to different catalogues for features and functions, for every catalogue:
how is it available?
how can we use it?
individual services for every catalogue?
links to DBpedia and other external sources
use of semantic web tools
faceted search
graphical representation (SNIK GUI)
use of external services

\section{Instance Tool}
\subsection{Ranking}
- problem of selecting amongst hundreds of entries
- combo box
- there are too many instances to comfortably fit a list
- add search
- rank them: (1) how to rank (2) how to save the precomputed ranking in a standardized way
- we use x by default but can be configured
- could use "dbo:wikiPageOutDegree" for operating system (in dbpedia live)
- but we just use the number of software products that support this operating system (dbo:operatingSystem) and sort by that descending and with a count of at least 3
- User Interface ontology by Tim Berners-Lee
\aurl{ui}{sortPriority}, highest first, using the number of times an operating system was used by a software system
or use negative numbers

\subsection{Similarity}
In order to check if a software product has already been modelled or if a similar alternative exists for a given product, we provide a similarity function between two given products. ...
DBpedia ranker?
- textual similarity between citations?
- jaccard between classifieds
- see "Ranking the Linked Data: The Case of DBpedia" \url{https://link.springer.com/content/pdf/10.1007\%2F978-3-642-13911-6.pdf}

\section{Related Work}
medfloss.org?

to read:\\
Ontology-based User Interface Development: User Experience Elements Pattern
Goals for a Human-Data Interface\footnote{\url{https://www.w3.org/DesignIssues/TabulatorGoals.html}}

\section{HTIO}
\subsection{Ontology}
\subsection{Data sources}

\section{Conclusions}\label{conclusions}
There is no longer \LaTeX{} example which was written by \cite{swo}.

\section{Future Work}
CaLiGraph\footnote{paper not yet finished, \url{http://caligraph.org}} offers a much larger amount of classes (\num{755964}) than DBpedia (685) provide additional knowledge by processing categories and listpages of Wikipedia using machine learning methods.

\bibliographystyle{plain}
\bibliography{paper}
\end{document}

