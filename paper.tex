\documentclass{article}

\usepackage{hyperref}
\usepackage[1000]{aurl}
\daurl{swo}{http://www.ebi.ac.uk/swo/}
\usepackage{graphicx}
\usepackage{microtype}
\usepackage{cleveref}
\usepackage{siunitx}
\usepackage{booktabs}
\usepackage{tabulary}
\usepackage{csquotes}
\usepackage{acronym}
\newcommand{\citet}{\cite}% citet is not defined without natbib
\newcommand{\citep}{\cite}% citep is not defined without natbib

\title{HITOLink: ontology-based creation of MedFloss software products}
\author{IMISE and UMIT people}
\date{\today}
\begin{document}
\maketitle

\begin{acronym}[SPARQL]
\acro{HITO}{Health IT Ontology}
\end{acronym}

\begin{abstract}
At the start, HITO defined everything as new classes and properties.
However this didn't take advantage of the LOD cloud and was thus changed.
Replacing self-defined classes with links to existing ones reduces development effort, improves maintainability and allows deeper exploration.
We investigate ontologies that are close to HITO for their suitability as replacements for HITO classes.
\end{abstract}

\section{Introduction}
\begin{figure}[h]
\includegraphics[width=\textwidth]{img/hito-diagram.pdf}
\end{figure}

\subsection{Subject}
The aim of \ac{HITO} is to enable systematic descriptions of application systems and software products in health IT.
We interlink and enrich \ac{HITO} using DBpedia, SWO and ...
The initial draft of \ac{HITO}~\citep{towardsprecise} defined everything as new classes and properties.

\begin{itemize}

\item Welche Situation liegt vor und was soll getan werden?
\item Worum geht es eigentlich?
\item In welcher Welt/Domäne oder welchem Arbeitsbereich/-gebiet bewegen wir uns im Rahmen der Arbeit

\end{itemize}

% \subsection{Problematik}
% \begin{itemize}
% \item Warum ist die geschilderte vorliegende Situation problematisch?
% \item Worin bestehen die Probleme?
% \item Für wen ist sie problematisch?
% \end{itemize}
%
% \subsection{Motivation}
%
% \begin{itemize}
%
% \item Warum lohnt es sich, die genannten Probleme zu lösen?
% \item Wer wird welchen Nutzen von dieser Abschlussarbeit haben?
% \item Warum ist die Arbeit wichtig?
% \item Wer wartet sehnlichst auf die Fertigstellung der Arbeit
%
% \end{itemize}
%
% \section{Problemstellung}
%
% Vermeiden Sie Formulierungen wie \glqq Es ist nicht bekannt, ob...\grqq oder \glqq Es existiert kein...\grqq.
%
% Solche Formulierungen kehren in der Regel einfach das bereits angedachte Lösungsmodell um und postulieren das Fehlen der angedachten Lösung einfach als Problem. Das ist ähnlich, wie wenn es  in der Werbung hieße ?Wenn Sie das Problem haben, dass Ihnen Aspirin fehlt, dann kaufen Sie doch Aspirin?. Sinnvoller ist diese Aussage:  ?Wenn Sie das Problem haben, dass Ihnen der Kopf weh tut, dann kaufen Sie doch Aspirin?. Es ist also bei der Problembeschreibung erforderlich, sich in die Lage dessen zu versetzen, den man mit der angedachten Lösung beglücken möchte. Sein Problem ist zu ermitteln und so zu formulieren, er/sie das Problem wiedererkennt und dadurch geneigt ist, sich für die Lösung des Problems zu interessieren.
%
% \begin{itemize}
%
% \item Welche der in der Problematik geschilderten Probleme sollen im Rahmen dieser Abschlussarbeit gelöst werden?
% \end{itemize}
% Bitte jedes Einzelproblem nummerieren und mit 1-2 Sätzen beschreiben:
%
% \begin{itemize}
%
%
% \item Problem P1: Problemname 1 mit kurzer Problembeschreibung
% \item Problem P2: Problemname 2 mit kurzer Problembeschreibung
% \end{itemize}
%
% \section{Zielsetzung}
%
% \begin{itemize}
% \item Welche Ergebnisse werden mit dieser Abschlussarbeit angestrebt und welche der o.g. Probleme sollen damit jeweils gelöst werden?
% \end{itemize}
% Bitte jedes Ziel kurz oder ggf. mit Stichworten beschreiben:
% \begin{itemize}
% \item Ziel(e)/angestrebte(s) Ergebnis(se) zur Lösung von Problem P1:
% 	\begin{itemize}
% 	\item Ziel Z1.1: ....
% 	\item Ziel Z1.2: ....
% 	\end{itemize}
% \end{itemize}
%
% \section{Aufgabenstellung}
%
% \begin{itemize}
% \item Wie sollen die o.g. Ziele erreicht werden?
% \item Was soll zur Erreichung der Ziele bzw. zur Schaffung der Ergebnisse getan werden?
% \item Welche Fragen müssen zur Erreichung der Ziele bzw. zur Schaffung der Ergebnisse beantwortet  werden?
% \end{itemize}
%
%
% Bitte geben Sie zu jedem der o.g. Ziele mindestens zwei Aufgaben bzw. Fragen an, die bearbeitet bzw. beantwortet werden sollen. Bitte jede Aufgabe bzw. Frage kurz oder ggf. mit Stichworten beschreiben:
%
% \begin{itemize}
% \item Aufgaben zu Ziel Z1.1:
% 	\begin{itemize}
% 	\item Aufgabe A1.1.1: ....
% 	\item Aufgabe A1.1.2: ....
% 	\end{itemize}
% \end{itemize}
%%%
\section{Related Ontologies}
\begin{itemize}
\item DBpedia~\citep{dbpedia}, the central knowledge base of the LOD cloud, contains many properties and classes related to software products. However the data quality is lacking.
\item SWO (The Software Ontology)~\citep{}
\item DOAP (Description of a Project)\footnote{\url{http://usefulinc.com/ns/doap}}

\end{itemize}


\subsection{Software License}

\begin{tabular}{llcc}
\toprule
Ontology	&property			&range	&instances\\
\midrule
DBpedia		&\aurl{dbo}{license}		&---	&---\\
SWO		&\aurl{swo}{SWO\_0000002}	&	&\\
DOAP		&\\
\bottomrule
\end{tabular}

\subsection{Operating System}

\begin{tabular}{llcc}
\toprule
Ontology	&property			&range	&instances\\
\midrule
DBpedia		&\aurl{dbo}{operatingSystem}	&---	&---\\
YAGO		&				&	&\\
SWO		&\\
DOAP		&\\
\bottomrule
\end{tabular}

\begin{tabular}{llcccc}
\toprule
Ontology	&class					&instances	&missing			&wrong\\
\midrule
YAGO		&\aurl{yago}{OperatingSystem106568134}	&320		&winxp, arch linux		&many\\
YAGO		&\aurl{yago}{WikicatOperatingSystems}	&		&winxp, arch linux		&many\\
SWO		&\\
DOAP		&\\
\bottomrule
\end{tabular}


\subsection{Programming Language}

\section{Other}
%\begin{}
link to studies, which in turn are systematically described by our ontology,
links to different catalogues for features and functions, for every catalogue:
how is it available?
how can we use it?
individual services for every catalogue?
links to DBpedia and other external sources
use of semantic web tools
faceted search
graphical representation (SNIK GUI)
use of external services

\section{Instance Tool}
\subsection{Ranking}
- problem of selecting amongst hundreds of entries
- combo box
- there are too many instances to comfortably fit a list
- add search
- rank them: (1) how to rank (2) how to save the precomputed ranking in a standardized way
- we use x by default but can be configured
- could use "dbo:wikiPageOutDegree" for operating system (in dbpedia live)
- but we just use the number of software products that support this operating system (dbo:operatingSystem) and sort by that descending and with a count of at least 3
- User Interface ontology by Tim Berners-Lee
\aurl{ui}{sortPriority}, highest first, using the number of times an operating system was used by a software system
or use negative numbers

\subsection{Similarity}
In order to check if a software product has already been modelled or if a similar alternative exists for a given product, we provide a similarity function between two given products. ...
DBpedia ranker?
- textual similarity between citations?
- jaccard between classifieds
- see "Ranking the Linked Data: The Case of DBpedia" \url{https://link.springer.com/content/pdf/10.1007\%2F978-3-642-13911-6.pdf}

\section{Related Work}
to read:\\
Ontology-based User Interface Development: User Experience Elements Pattern
Goals for a Human-Data Interface\footnote{\url{https://www.w3.org/DesignIssues/TabulatorGoals.html}}

\section{Conclusions}\label{conclusions}
There is no longer \LaTeX{} example which was written by \cite{swo}.

\section{Future Work}
CaLiGraph\footnote{paper not yet finished, \url{http://caligraph.org}} offers a much larger amount of classes (\num{755964}) than DBpedia (685) provide additional knowledge by processing categories and listpages of Wikipedia using machine learning methods.

\bibliographystyle{plain}
\bibliography{paper}
\end{document}
