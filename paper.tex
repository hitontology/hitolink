\documentclass{article}

\usepackage{hyperref}
\usepackage[1000]{aurl}
\daurl{swo}{http://www.ebi.ac.uk/swo/}
\usepackage{graphicx}
\usepackage{microtype}
\usepackage{cleveref}
\usepackage{siunitx}
\usepackage{booktabs}
\usepackage{tabulary}
\usepackage{csquotes}
\newcommand{\citet}{\cite}% citet is not defined without natbib
\newcommand{\citep}{\cite}% citep is not defined without natbib

\title{ Sustainability of HITO by using Linked Open Data Technology in order to reuse existing catalogues and to make  HITO results reusable by other projects}
\author{IMISE and UMIT people}
\date{\today}
\begin{document}
\maketitle

\begin{abstract}
At the start, HITO defined everything as new classes and properties.
However this didn't take advantage of the LOD cloud and was thus changed.
Replacing self-defined classes with links to existing ones reduces development effort, improves maintainability and allows deeper exploration.
We investigate ontologies that are close to HITO for their suitability as replacements for HITO classes.
\end{abstract}

\section{Introduction}
\begin{figure}[h]
\includegraphics[width=\textwidth]{img/hito-diagram.pdf}
\end{figure}

\section{Related Ontologies}
\begin{itemize}
\item DBpedia~\citep{dbpedia}, the central knowledge base of the LOD cloud, contains many properties and classes related to software products. However the data quality is lacking.
\item SWO (The Software Ontology)~\citep{}
\item DOAP (Description of a Project)\footnote{\url{http://usefulinc.com/ns/doap}}

\end{itemize}


\subsection{Software License}

\begin{tabular}{llcc}
\toprule
Ontology	&property			&range	&instances\\
\midrule
DBpedia		&\aurl{dbo}{license}		&---	&---\\
SWO		&\aurl{swo}{SWO\_0000002}	&	&\\
DOAP		&\\
\bottomrule
\end{tabular}

\subsection{Operating System}

\begin{tabular}{llcc}
\toprule
Ontology	&property			&range	&instances\\
\midrule
DBpedia		&\aurl{dbo}{operatingSystem}	&---	&---\\
YAGO		&				&	&\\
SWO		&\\
DOAP		&\\
\bottomrule
\end{tabular}

\begin{tabular}{llcccc}
\toprule
Ontology	&class					&instances	&missing			&wrong\\
\midrule
YAGO		&\aurl{yago}{OperatingSystem106568134}	&320		&winxp, arch linux		&many\\
YAGO		&\aurl{yago}{WikicatOperatingSystems}	&		&winxp, arch linux		&many\\
SWO		&\\
DOAP		&\\
\bottomrule
\end{tabular}


\subsection{Programming Language}

\section{Other}
%\begin{}
link to studies, which in turn are systematically described by our ontology,
links to different catalogues for features and functions, for every catalogue:
how is it available?
how can we use it?
individual services for every catalogue?
links to DBpedia and other external sources
use of semantic web tools
faceted search
graphical representation (SNIK GUI)
use of external services

\section{Instance Tool}
\subsection{Ranking}
- problem of selecting amongst hundreds of entries
- combo box
- there are too many instances to comfortably fit a list
- add search
- rank them: (1) how to rank (2) how to save the precomputed ranking in a standardized way
- we use x by default but can be configured
- could use "dbo:wikiPageOutDegree" for operating system (in dbpedia live)
- but we just use the number of software products that support this operating system (dbo:operatingSystem) and sort by that descending and with a count of at least 3
- User Interface ontology by Tim Berners-Lee
\aurl{ui}{sortPriority}, highest first, using the number of times an operating system was used by a software system
or use negative numbers

\subsection{Similarity}
In order to check if a software product has already been modelled or if a similar alternative exists for a given product, we provide a similarity function between two given products. ...
DBpedia ranker?
- textual similarity between citations?
- jaccard between classifieds
- see "Ranking the Linked Data: The Case of DBpedia" \url{https://link.springer.com/content/pdf/10.1007\%2F978-3-642-13911-6.pdf}

\section{Related Work}
to read:\\
Ontology-based User Interface Development: User Experience Elements Pattern
Goals for a Human-Data Interface\footnote{\url{https://www.w3.org/DesignIssues/TabulatorGoals.html}}

\section{Conclusions}\label{conclusions}
There is no longer \LaTeX{} example which was written by \cite{swo}.

\section{Future Work}
CaLiGraph\footnote{paper not yet finished, \url{http://caligraph.org}} offers a much larger amount of classes (\num{755964}) than DBpedia (685) provide additional knowledge by processing categories and listpages of Wikipedia using machine learning methods.

\bibliographystyle{plain}
\bibliography{paper}
\end{document}
